\documentclass{article}

\usepackage[T2A]{fontenc} 
\usepackage[utf8]{inputenc} 
\usepackage[english,russian]{babel}
\usepackage{graphicx}
\usepackage{amsmath} 
\usepackage{amsfonts} 
\usepackage{titlesec} 
\usepackage{titling} 
\usepackage{geometry} 
\usepackage{pgfplots}
\pgfplotsset{compat=1.9}

\titleformat{\section}
  {\normalfont\Large\bfseries}{\thesection}{1em}{}
\titleformat{\subsection}
  {\normalfont\large\bfseries}{\thesubsection}{1em}{}


\setlength{\droptitle}{-6em} 
\title{\vspace{-2cm}Интерполяционная формула Стирлинга через центральные разности}
\author{Винницкая Дина Сергеевна}
\date{Группа: Б9122-02-03-01сцт}


\geometry{a4paper, margin=2cm}

\maketitle

\begin{document}

\section*{ Вывод формулы}

Интерполяционная формула Стирлинга через центральные разности выражается следующим образом:

Пусть дана функция \( f(x) \) и узлы интерполяции \( (x_0, x_1, x_2, ..., x_n) \) с равным шагом h. Центральные разности для интерполяционной формулы Стирлинга определяются как:

\[ \Delta^2 f(x_i) = f(x_{i+1}) - 2f(x_i) + f(x_{i-1}). \]

Интерполяционная формула Стирлинга имеет вид:

\[ f(x) \approx f(x_i) + p_1(x-x_i) + p_2(x-x_i)(x-x_{i-1}), \]

где

\[ p_1 = \frac{\Delta f(x_i)}{h} \]
\[ p_2 = \frac{\Delta^2 f(x_i)}{2h^2}. \]

Таким образом, интерполяционная формула Стирлинга через центральные разности будет записана следующим образом:

\[ f(x) \approx f(x_i) + \frac{\Delta f(x_i)}{h} (x-x_i) + \frac{\Delta^2 f(x_i)}{2h^2} (x-x_i)(x-x_{i-1}). \]

Это позволяет интерполировать функцию в точке x с помощью значений функции и ее производных в узлах интерполяции.

\section*{ Когда предпочтительнее пользоваться ею, и почему?}
Интерполяционная формула Стирлинга через центральные разности применяется в тех случаях, когда необходимо интерполировать функцию по равномерно распределенным узлам с использованием центральных разностей. Ее предпочтительно использовать в следующих случаях:

\begin{enumerate}
    \item Равномерные узлы: Формула Стирлинга хорошо подходит для интерполяции на равномерно распределенных узлах, где шаг между узлами одинаковый.
    \item Центральные разности: Если доступны значения функции и ее производных в узлах, то формула Стирлинга через центральные разности позволяет учесть информацию о поведении функции вблизи интерполируемой точки.

    \item Учет квадратичной зависимости: Использование квадратичного члена в формуле Стирлинга позволяет более точно аппроксимировать функцию в окрестности узлов, что особенно полезно при необходимости более точной интерполяции.
    \item Простота вычислений: Формула Стирлинга через центральные разности может быть более простой в вычислениях по сравнению с некоторыми другими методами интерполяции, что делает ее привлекательным выбором в случаях, когда требуется быстрая оценка значений функции между узлами.
    
\end{enumerate}

Таким образом, интерполяционная формула Стирлинга через центральные разности является хорошим выбором для задач интерполяции на равномерно распределенных узлах с известными значениями функции и ее производных в узлах.

\end{document}

